% Codificación: UTF-8
%
% Autor: Sebastian Sepulveda A.
%        Facultad de Ciencias Físicas y Matemáticas
%        Universidad de Chile
%        sesepulveda@ug.uchile.cl

\documentclass[letterpaper,12pt]{article}
% ======================= Preambulo =======================
\usepackage[utf8]{inputenc}
\usepackage[spanish,es-tabla]{babel}
\usepackage{blindtext}
\usepackage{imakeidx}
\usepackage{multicol}
\usepackage{anysize}
\usepackage{amsmath}
\usepackage{amsfonts}
\usepackage[colorlinks=true,linkcolor=magenta]{hyperref}
\usepackage[usenames,dvipsnames,svgnames,table]{xcolor}
\usepackage{listings}
\usepackage[symbols,
  stylemods={tree},
  record 
]{glossaries-extra}
%para glosarios consultar bibliografia


%Se le asigna el titulo a los grupos del glosario
\glsxtrsetgrouptitle{greek}{Grupo 1}
\glsxtrsetgrouptitle{latin}{Grupo 2}

\GlsXtrLoadResources[
 src={grupo1},
 type=symbols,
 group={greek},% assign group label
 set-widest,% needed for 'alttree' styles
 save-locations=false
]

\GlsXtrLoadResources[
 src={grupo2},
 type=symbols,
 group={latin},% assign group label
 set-widest,% needed for 'alttree' styles
 save-locations=false
]

\marginsize{2cm}{2cm}{1cm}{2cm}
%Titulo y agregados
\title{\Huge\bfseries Documentación \texttt{Vpython}}
\author{\Large FI2001-3 Mecánica 2018\\ Universidad de Chile}
\date{Junio}

% ======================= INICIO DEL DOCUMENTO =======================
\begin{document}

% ======================= PORTADA pag 1=======================
\pagestyle{empty}
\maketitle
\begin{abstract}
El grupo de ``Coordenadas uwu'' surgió el año 2018 con la inciativa del profesor Alvaro Nuñez, junto con el apoyo del estudiante Rodrigo Jaeschke, para desarrollar animaciones de:
\begin{itemize}
\item Sistemas de coordenadas, vectores unitarios, curvas de coordenadas constante, etc
\item Sistemas mecánicos sencillos, ilustrando coordenadas intrinsecas, velocidades, aceleraciones, etc
\item Sistemas más complejos... flujo de energía, deformaciones elásticas, etc. 
\end{itemize}
El grupo es conformado actualmente por estudiantes de segundo año de la FCFM, y está divididos en 3 grupos de trabajos:
\begin{itemize}
	\item Grupo 1: Vale G., Palo V. y Martin M.
	\item Grupo 2: Boris C., Seba S. y Bastián F.
	\item Grupo 3: Bruno R., Tomás R y Feña S.
\end{itemize}
\end{abstract}
\thispagestyle{empty} % no numera

% ======================= TABLA DE CONTENIDOS pag 2 =======================
\newpage
\thispagestyle{empty}
\tableofcontents
%Tablas
\listoftables
%Figuras
\listoffigures
%Glosario
\printunsrtglossary[type=symbols,style=alttreegroup,title={Lista de simbolos}]
\pagenumbering{arabic}
% ======================= otros pag 3-- =======================
\newpage
Para el proyecto vamos a usar python 3.6 y Vpython. Para descargar python 3.6 y vpython se recomienda descargar anaconda (es un paquete en el que viene python y muuuchos paquetes muy útiles) para python. 

%EDITAR
\section{Gravitación}
Explicación de simbolos que se ocuparán en los programas:

Simbolos griegos: 
 
Grupo 2: $\gls{chi}$, $\gls{alpha}$, $\gls{zeta}$, $\gls{lambda}$.

Grupo 1: $\gls{x}$, $\gls{v}$, $\gls{a}$, $\gls{t}$,
$\gls{F}$.

%EDITAR
\begin{itemize}
    \item Interpretado $\gls{delta}$,
    \item Indentación obligatoria
    \item Distingue mayúsculas - minúsculas
    \item No hay declaración de variables (\textit{dynamic typing})
    \item Orientado a objetos  
    \item Garbage colector: quita los objetos a los que no haga referencia nada
\end{itemize}

\section{Cinematica}



\newpage
\begin{thebibliography}{X}
	\bibitem{vpython} \textsc{Sitio web Vpython}\ --	\textit{Documentación y más}\ -- 	\url{http://vpython.org/}

	\bibitem{glowscript} \textsc{Glowscript} \ --	\textit{Almacén de documentos, online}\ --  	\url{http://www.glowscript.org/}

	\bibitem{githubgrupo} \textsc{Grupo en Github}\ --  	\textit{Archivos del grupo, Github}\ -- 	\url{https://github.com/vpythonfcfm}

	\bibitem{gitlabgrupo} \textsc{Grupo en GitLab}\ --  	\textit{Archivos del grupo, GitLab}\ -- 	\url{https://gitlab.com/mecanica}

	\bibitem{bdrive} \textsc{Drive}\\
	\textit{Documentación en drive}\\
	\url{https://drive.google.com/drive/folders/1NtJFtAmzxQd_YRu3f68Fnss4KQNrH-Un}

	\bibitem{beneficios} \textsc{Programas utiles para desarrollar mejor los códigos}\\
	\textit{Beneficios alumnos Uchile}\\
	\url{https://www.u-cursos.cl/usuario/77a5152ba2963e5296264676485b1c05/mi_blog/o/23555}

	\bibitem{Glosarios} \textsc{Principal de glosarios}\\
	\textit{Programacion lineal y flujo en redes}, segunda edicion\\
	\url{http://mirrors.ibiblio.org/CTAN/macros/latex/contrib/glossaries-extra/glossaries-extra-manual.pdf}

	\bibitem{Glosarios2} \textsc{Principal de glosarios} y \textsc{H.D. Sherali}\\
	\textit{Programacion lineal y flujo en redes}, segunda edicion\\
	\url{https://es.sharelatex.com/learn/Lists_of_tables_and_figures}

	\bibitem{Glosarios3} \textsc{Principal de glosarios} y \textsc{H.D. Sherali}\\
	\textit{Programacion lineal y flujo en redes}, segunda edicion\\
	\url{https://tex.stackexchange.com/questions/255787/how-to-index-figures?utm_medium=organic&utm_source=google_rich_qa&utm_campaign=google_rich_qa}

	\bibitem{Glosarios4} \textsc{Principal de glosarios} y \textsc{H.D. Sherali}\\
	\textit{Programacion lineal y flujo en redes}, segunda edicion\\
	\url{https://tex.stackexchange.com/questions/348640/how-to-effectively-use-list-of-symbols-for-a-thesis?utm_medium=organic&utm_source=google_rich_qa&utm_campaign=google_rich_qa}

	\bibitem{Glosarios5} \textsc{Principal de glosarios} y \textsc{H.D. Sherali}\\
	\textit{Programacion lineal y flujo en redes}, segunda edicion\\
	\url{https://es.sharelatex.com/learn/Glossaries}

	\bibitem{Glosarios6} \textsc{Principal de glosarios} y \textsc{H.D. Sherali}\\
	\textit{Programacion lineal y flujo en redes}, segunda edicion\\
	\url{https://github.com/nlct/bib2gls}

	\bibitem{Glosarios7} \textsc{Principal de glosarios} y \textsc{H.D. Sherali}\\
	\textit{Programacion lineal y flujo en redes}, segunda edicion\\
	\url{https://ctan.org/pkg/bib2gls}

\end{thebibliography}

\end{document}
